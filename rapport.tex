\documentclass[french]{report}
\usepackage[utf8]{inputenc}
\usepackage[T1]{fontenc}
\usepackage{lmodern}
\usepackage[a4paper]{geometry}
\usepackage{babel}

%% Pour les images
\usepackage{graphicx}
\usepackage{subfigure} %permet un meilleur affichage d'une image par rapport à une autre
\usepackage{xcolor} %permet d'éviter les conflicts de paquet entre color et xcolor
\renewcommand\thesubfigure{} %pour retirer la numérotation des caption sur les sous figures
\renewcommand\thesubtable{}

%% contrôler mes hauts de page !
\usepackage{fancyhdr} %\usepackage{fancyheadings}

%pour personnaliser les listes
\usepackage{enumerate}

%% pour les url
\usepackage{hyperref} % exemple d'utilisation: \href{www.google.com}{toto}

%% pour les inclusions de fichiers pdf
\usepackage[final]{pdfpages} 

%% Pour les couleurs
%\usepackage{colortbl} %dans les tableaux

%% Définition des newcommand
%% %%%%%%%%%%%%%%%%%%%%%%%%%%%% %
%% INFORMATIONS SUR LE DOCUMENT
%% ---------------------------- %
%% But: Ce fichier contient les 
%% différents accronymes, macro 
%% ...
%% %%%%%%%%%%%%%%%%%%%%%%%%%%%% %


%%%%%%%%%%%%%%%%%%%%%%%%%%%%%%%%%%%%%%%%%%%%%%%%%%%%%%%%%% Les principaux acteurs du projet
\newcommand{\etudiantJP}{Jonathan Poncy}
\newcommand{\etudiantRD}{Romain Daquin}
\newcommand{\etudiantSL}{Stéphane Legrand}
\newcommand{\civiliteResponsableProet}{Monsieur}
\newcommand{\responsableProjet}{Éric Ramat}
\newcommand{\responsableDesProjets}{Julien Dehos}
\newcommand{\pepit}{\href{pepit.be}{Pepit}}
\newcommand{\pepitSite}{\href{http://pepit.be}{pepit.be}}
\newcommand{\pepitMobil}{Pépit Mobil}

%%%%%%%%%%%%%%%%%%%%%%%%%%%%%%%%%%%%%%%%%%%%%%%%%%%%%%%%%% macros techniques
\newcommand{\android}{\href{http://fr.wikipedia.org/wiki/Android}{\textbf{A}ndroid}}
\newcommand{\market}{\gp}
\newcommand{\gp}{\href{http://fr.wikipedia.org/wiki/Google_Play}{Google Play}}
\newcommand{\google}{Google}
\newcommand{\googleDrive}{\google{} Drive}
\newcommand{\os}{\href{http://fr.wikipedia.org/wiki/Syst\%C3\%A8me_d'exploitation}{Système d'Exploitation}}
\newcommand{\eclipse}{\href{http://fr.wikipedia.org/wiki/Eclipse_(logiciel)}{eclipse}}
\newcommand{\ide}{\href{http://fr.wikipedia.org/wiki/Environnement_de_d\%C3\%A9veloppement_int\%C3\%A9gr\%C3\%A9}{I.D.E.}}
\newcommand{\linux}{\href{http://fr.wikipedia.org/wiki/Linux}{Linux}}
\newcommand{\java}{\href{http://fr.wikipedia.org/wiki/Java_(langage)}{Java}}
\newcommand{\github}{\href{http://fr.wikipedia.org/wiki/GitHub}{github}}
\newcommand{\ubuntu}{ubuntu}
\newcommand{\plugin}{plugin}
\newcommand{\plugins}{plugins}

%%%%%%%%%%%%%%%%%%%%%%%%%%%%%%%%%%%%%%%%%%%%%%%%%%%%%%%%%% termes Agiles
\newcommand{\sprint}{\href{http://fr.wikipedia.org/wiki/Scrum\_(m\%C3\%A9thode)\#Le_Sprint}{Sprint}}
\newcommand{\agile}{Agile}%oui, avec un 'A'
\newcommand{\scrum}{\href{http://fr.wikipedia.org/wiki/Scrum\_(m\%C3\%A9thode)}{Scrum}}

%%%%%%%%%%%%%%%%%%%%%%%%%%%%%%%%%%%%%%%%%%%%%%%%%%%%%%%%%% liens
\newcommand{\dossierpdf}{documents_externes/}
\newcommand{\dossierimages}{images/}
%\newcommand{\}{}



%% style des titres
\usepackage{sectsty}
%\convertcolorspec{HTML}{eda299}{RGB}{\toto} pour recuperer un code RGB
\definecolor{color_chapter}{RGB}{0,0,128}
\definecolor{color_section}{RGB}{112,147,219}
\definecolor{color_subsection}{RGB}{19,81,204}
\chapterfont{\color{color_chapter}{}\fontseries{b}}
\sectionfont{\color{blue}{}\fontseries{b}}
\subsectionfont{\color{color_subsection}{}\fontseries{b}}
\subsubsectionfont{\color{cyan}{}\fontseries{b}}


%% %%%%%%%%%%%%%%%%%%%%%%%%%%%% %
%% INFORMATIONS SUR LE DOCUMENT
%% ---------------------------- %
%% ??? indique un texte à changer
%%
%% tâches (spéciales) à réaliser:
%%  l'ensemble des tâches seront
%%  désormés décrites sur github
%%  sous forme "d'issues"
%%
%% %%%%%%%%%%%%%%%%%%%%%%%%%%%% %
\begin{document}
%% page de titre

\begin{titlepage}

\begin{center}


%% partie supérieur de la page
% TODO Mettre une image pour le projet
\includegraphics[width=16cm]{images/pepit-logo}\\[1cm]    

\textsc{\LARGE \textbf{U}niversité du \textbf{L}ittoral \textbf{C}ôte d'\textbf{O}pale}\\[1.5cm]
\textsc{\Large 2\ieme{} année de master informatique}\\[0.5cm]


% Titre
%\HRule \\[0.4cm]
{ \huge \bfseries Rapport de stage}\\[0.4cm]

%\HRule \\[1.5cm]

% Auteurs et superviseurs
\begin{minipage}{0.3\textwidth}
\begin{flushleft} \large
\emph{Étudiants:}\\
Jonathan \textsc{Poncy}
Romain \textsc{Daquin}
\end{flushleft}
\end{minipage}
%\begin{minipage}{0.3\textwidth}
%\begin{center} \large
%\emph{Responsable des Projets:} \\
%\responsableDesProjets
%\end{center}
%\end{minipage}
\begin{minipage}{0.3\textwidth}
\begin{flushright} \large
\emph{Encadrant} \\
\responsableProjet
\end{flushright}
\end{minipage}

\vfill

% Bottom of the page
%{\large \today}
		\begin{footnotesize}
			\begin{tabular}{p{10cm} r}
			\includegraphics[width=3cm]{images/logo_ulco}
				& ??? logo pépit \\
			Université de Littoral Côte d'Opale
				& Pépit ??? \\
			50 rue Ferdinand Buisson 
				& ??? \\
			62100 Calais
				& ??? \\
			Tél : +33 (0)3 21 46 36 00
				& ??? \\
			Fax : +33 (0)3 21 46 36 69
				& ??? \\
			\end{tabular}
		\end{footnotesize}
\end{center}

\end{titlepage}
\newpage
\null
\newpage

%\maketitle

%% hauts de page :
\pagestyle{fancyplain}
\renewcommand{\chaptermark}[1]{\markboth{\chaptername\ \thechapter. #1}{}}
\renewcommand{\sectionmark}[1]{\markright{\thesection. #1}}
\lhead[]{\fancyplain{}{\bfseries\leftmark}}
\rhead[]{\fancyplain{}{\bfseries\thepage}}
\cfoot{}
% 
%
\chapter*{Remerciements}
Nous tenons tout dabord à remercier \civiliteResponsableProet{} \responsableProjet{} pour ???
%ne pas oublir de remercier les membre de l'équipe de pepit nous ayant aidé
% le relecteur de la fac
% ???


%% ************************************************** %
\section*{Distribution}
\begin{description}
\item [Responsable des projets:] \responsableDesProjets
\item [Responsable de projet:] \responsableProjet
%\item [Membres du jury:]  ???
\item [Auteurs:] \etudiantJP, \etudiantRD
\item [Autre membre du projet:] \etudiantSL
\end{description}


%% table des matières
\tableofcontents


%%%%%%%%%%%%%%%%%%%%%%%%%%%%%%%%%%%%%%%%%%%%%%%%%%%%%%%%%% Introduction :
%\chapter*{Introduction}
%\addcontentsline{toc}{chapter}{Introduction} %Permet de rajouter l'intro dans la table des matière


%%%%%%%%%%%%%%%%%%%%%%%%%%%%%%%%%%%%%%%%%%%%%%%%%%%%%%%%%% Chap :
\chapter{Présentation du projet}
???
%% ************************************************** %
\section{\pepit}
???
%% ************************************************** %
\section{Et nous, que vient-on faire là dedans ?}
???
\subsection{But du projet}
???


%%%%%%%%%%%%%%%%%%%%%%%%%%%%%%%%%%%%%%%%%%%%%%%%%%%%%%%%%%% Chap :
\chapter{Ressources et outils utilisés}
%% ************************************************** %
\section{Documentation}
\subsection{\googleDrive}
% TODO rajouter le lien vers l'annexe
Il a été décidé lors de notre première réunion (voir annexe ???/???) que les documents de travails (exemple: résumé de réunion, glossaire, ...) sont a stocker dans un repertoire de \googleDrive{}. Les document de type \og{}\google{}\fg{} sont à privilégier.
\subsection{Latex}
Nous nous sommes servi du langage \LaTeX{} uniquement pour la rédaction de ce rapport et pour notre présentation.
%% ************************************************** %
\section{Développement}
%Pour développer des applications \android{}, \google{} préconise l'utilisation d'\eclipse{} sur \ubuntu{}.
\begin{description}
\item[\os{}: ] Le développement se fait sur différentes distributions \linux{}
\item[\ide{} et language: ] Pour le développement, nous utilisons le SDK de \android{} couplé avec l'\ide{} \eclipse{}. Le language utilisé est \java{}
\item[Gestionnaire de depôts: ] Ce projet étant réalisé en équipe qui ne se voit pas régulièrement. Il a été décidé dans un premier temps (par \responsableProjet{}) d'utiliser un gestionnaire de source. Nous utilisons \github{} à la fois pour les sources du projet mais aussi (dans un second projet \github{}) les sources de notre rapport
\end{description}


%%%%%%%%%%%%%%%%%%%%%%%%%%%%%%%%%%%%%%%%%%%%%%%%%%%%%%%%%%% Chap :
%\chapter{à compléter}
%%% ************************************************** %
%\section{à compléter}


%%%%%%%%%%%%%%%%%%%%%%%%%%%%%%%%%%%%%%%%%%%%%%%%%%%%%%%%%% Chap :
\chapter*{Bilan et conclusion}
%% ************************************************** %
\section*{Bilan}
???
%% ************************************************** %
\section*{Conclusion}
???


%\appendix
\end{document}

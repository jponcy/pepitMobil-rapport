\documentclass[french]{report}
\usepackage[utf8]{inputenc}
\usepackage[T1]{fontenc}
\usepackage{lmodern}
\usepackage[a4paper]{geometry}
\usepackage{babel}

%% Pour les images
\usepackage{graphicx}
\usepackage{subfigure} %permet un meilleur affichage d'une image par rapport à une autre
\usepackage{xcolor} %permet d'éviter les conflicts de paquet entre color et xcolor
\renewcommand\thesubfigure{} %pour retirer la numérotation des caption sur les sous figures
\renewcommand\thesubtable{}

%% contrôler mes hauts de page !
\usepackage{fancyhdr} %\usepackage{fancyheadings}

%pour personnaliser les listes
\usepackage{enumerate}

%% pour les url
\usepackage{hyperref} % exemple d'utilisation: \href{www.google.com}{toto}

%% pour les inclusions de fichiers pdf
\usepackage[final]{pdfpages} 

%% Pour les couleurs
%\usepackage{colortbl} %dans les tableaux

%% Définition des newcommand
%% %%%%%%%%%%%%%%%%%%%%%%%%%%%% %
%% INFORMATIONS SUR LE DOCUMENT
%% ---------------------------- %
%% But: Ce fichier contient les 
%% différents accronymes, macro 
%% ...
%% %%%%%%%%%%%%%%%%%%%%%%%%%%%% %


%%%%%%%%%%%%%%%%%%%%%%%%%%%%%%%%%%%%%%%%%%%%%%%%%%%%%%%%%% Les principaux acteurs du projet
\newcommand{\etudiantJP}{Jonathan Poncy}
\newcommand{\etudiantRD}{Romain Daquin}
\newcommand{\etudiantSL}{Stéphane Legrand}
\newcommand{\civiliteResponsableProet}{Monsieur}
\newcommand{\responsableProjet}{Éric Ramat}
\newcommand{\responsableDesProjets}{Julien Dehos}
\newcommand{\pepit}{\href{pepit.be}{Pepit}}
\newcommand{\pepitSite}{\href{http://pepit.be}{pepit.be}}
\newcommand{\pepitMobil}{Pépit Mobil}

%%%%%%%%%%%%%%%%%%%%%%%%%%%%%%%%%%%%%%%%%%%%%%%%%%%%%%%%%% macros techniques
\newcommand{\android}{\href{http://fr.wikipedia.org/wiki/Android}{\textbf{A}ndroid}}
\newcommand{\market}{\gp}
\newcommand{\gp}{\href{http://fr.wikipedia.org/wiki/Google_Play}{Google Play}}
\newcommand{\google}{Google}
\newcommand{\googleDrive}{\google{} Drive}
\newcommand{\os}{\href{http://fr.wikipedia.org/wiki/Syst\%C3\%A8me_d'exploitation}{Système d'Exploitation}}
\newcommand{\eclipse}{\href{http://fr.wikipedia.org/wiki/Eclipse_(logiciel)}{eclipse}}
\newcommand{\ide}{\href{http://fr.wikipedia.org/wiki/Environnement_de_d\%C3\%A9veloppement_int\%C3\%A9gr\%C3\%A9}{I.D.E.}}
\newcommand{\linux}{\href{http://fr.wikipedia.org/wiki/Linux}{Linux}}
\newcommand{\java}{\href{http://fr.wikipedia.org/wiki/Java_(langage)}{Java}}
\newcommand{\github}{\href{http://fr.wikipedia.org/wiki/GitHub}{github}}
\newcommand{\ubuntu}{ubuntu}
\newcommand{\plugin}{plugin}
\newcommand{\plugins}{plugins}

%%%%%%%%%%%%%%%%%%%%%%%%%%%%%%%%%%%%%%%%%%%%%%%%%%%%%%%%%% termes Agiles
\newcommand{\sprint}{\href{http://fr.wikipedia.org/wiki/Scrum\_(m\%C3\%A9thode)\#Le_Sprint}{Sprint}}
\newcommand{\agile}{Agile}%oui, avec un 'A'
\newcommand{\scrum}{\href{http://fr.wikipedia.org/wiki/Scrum\_(m\%C3\%A9thode)}{Scrum}}

%%%%%%%%%%%%%%%%%%%%%%%%%%%%%%%%%%%%%%%%%%%%%%%%%%%%%%%%%% liens
\newcommand{\dossierpdf}{documents_externes/}
\newcommand{\dossierimages}{images/}
%\newcommand{\}{}



%% style des titres
\usepackage{sectsty}
%\convertcolorspec{HTML}{eda299}{RGB}{\toto} pour recuperer un code RGB
%\definecolor{color_chapter}{RGB}{0,0,128}
%\definecolor{color_section}{RGB}{112,147,219}
%\definecolor{color_subsection}{RGB}{19,81,204}
%\chapterfont{\color{color_chapter}{}\fontseries{b}}
%\sectionfont{\color{blue}{}\fontseries{b}}
%\subsectionfont{\color{color_subsection}{}\fontseries{b}}
%\subsubsectionfont{\color{cyan}{}\fontseries{b}}


%% %%%%%%%%%%%%%%%%%%%%%%%%%%%% %
%% INFORMATIONS SUR LE DOCUMENT
%% ---------------------------- %
%% ??? indique un texte à changer
%%
%% tâches (spéciales) à réaliser:
%%  l'ensemble des tâches seront
%%  désormés décrites sur github
%%  sous forme "d'issues"
%%
%% %%%%%%%%%%%%%%%%%%%%%%%%%%%% %
\begin{document}
%% page de titre

\begin{titlepage}

\begin{center}


%% partie supérieur de la page
% TODO Mettre une image pour le projet
\includegraphics[width=16cm]{images/pepit-logo}\\[1cm]    

\textsc{\LARGE \textbf{U}niversité du \textbf{L}ittoral \textbf{C}ôte d'\textbf{O}pale}\\[1.5cm]
\textsc{\Large 2\ieme{} année de master informatique}\\[0.5cm]


% Titre
%\HRule \\[0.4cm]
{ \huge \bfseries Rapport de stage}\\[0.4cm]

%\HRule \\[1.5cm]

% Auteurs et superviseurs
\begin{minipage}{0.3\textwidth}
\begin{flushleft} \large
\emph{Étudiants:}\\
Jonathan \textsc{Poncy}
Romain \textsc{Daquin}
\end{flushleft}
\end{minipage}
%\begin{minipage}{0.3\textwidth}
%\begin{center} \large
%\emph{Responsable des Projets:} \\
%\responsableDesProjets
%\end{center}
%\end{minipage}
\begin{minipage}{0.3\textwidth}
\begin{flushright} \large
\emph{Encadrant} \\
\responsableProjet
\end{flushright}
\end{minipage}

\vfill

% Bottom of the page
%{\large \today}
		\begin{footnotesize}
			\begin{tabular}{p{10cm} r}
			\includegraphics[width=3cm]{images/logo_ulco}
				& ??? logo pépit \\
			Université de Littoral Côte d'Opale
				& Pépit ??? \\
			50 rue Ferdinand Buisson 
				& ??? \\
			62100 Calais
				& ??? \\
			Tél : +33 (0)3 21 46 36 00
				& ??? \\
			Fax : +33 (0)3 21 46 36 69
				& ??? \\
			\end{tabular}
		\end{footnotesize}
\end{center}

\end{titlepage}
\newpage
\null
\newpage

%\maketitle

%% hauts de page :
\pagestyle{fancyplain}
\renewcommand{\chaptermark}[1]{\markboth{\chaptername\ \thechapter. #1}{}}
\renewcommand{\sectionmark}[1]{\markright{\thesection. #1}}
\lhead[]{\fancyplain{}{\bfseries\leftmark}}
\rhead[]{\fancyplain{}{\bfseries\thepage}}
\cfoot{}
% 
%
\chapter*{Remerciements}
\paragraph{}En premier lieu nous tenons à remercier \civiliteResponsableProet{} \responsableProjet{}, responsable du projet \pepitMobil, pour son suivit au quotidien ainsi que son aide précieuse tout au long du projet.
\paragraph{}Merci également à \etudiantSL, collaborateur du projet \pepitMobil.
\paragraph{}Nous remercions l'équipe du projet Pepit, pour leur disponibilité.
\paragraph{}Un dernier remerciement pour Arnaud Lewandowski, qui a été notre rapport tout au long de ce projet.


%% ************************************************** %
\section*{Distribution}
\begin{description}
\item [Responsable des projets:] \responsableDesProjets
\item [Responsable de projet:] \responsableProjet
%\item [Membres du jury:]  ???
\item [Auteurs:] \etudiantJP, \etudiantRD
\item [Autre membre du projet:] \etudiantSL
\end{description}


%% table des matières
\tableofcontents


%%%%%%%%%%%%%%%%%%%%%%%%%%%%%%%%%%%%%%%%%%%%%%%%%%%%%%%%%% Introduction :
%\chapter*{Introduction}
%\addcontentsline{toc}{chapter}{Introduction} %Permet de rajouter l'intro dans la table des matière


%%%%%%%%%%%%%%%%%%%%%%%%%%%%%%%%%%%%%%%%%%%%%%%%%%%%%%%%%% Chap :
\chapter{Présentation du projet}
\paragraph{}Pour cette deuxième année de Master Informatique, nous avons comme chaque année un projet à réaliser. Actuellement en Master option de l'Ingénierie du logiciel libre, nous avons choisi un projet qui est basé sur le logiciel libre. Celui-ci est sous la responsabilité de Mr \responsableProjet.
%% ************************************************** %
\section{\pepit}
\paragraph{}PEPIT est né le 15 mai 2004 à Mouscron en Belgique. Il s'agissait, à l'époque, de créer une plateforme numérique d'échange d'exercices entre 5 écoles primaires et le premier degré d'une école secondaire.
\paragraph{}Le projet a connu différentes évolutions au fil du temps, la troisième version du site est actuellement en ligne. Cette dernière version contient plus de 1250 exercices en ligne, la variété et la qualité de nos exercices sont appréciés les mois "scolaires" par plus de 200.000 "visiteurs uniques".
%% ************************************************** %
\section{Et nous, que vient-on faire là dedans ?}
\paragraph{}Actuellement, le pepit.be est un portail d'accès à des applications Flash, où il est possible de jouer directement sur le site ou de télécharger le jeu sur l'ordinateur (PC ou MAC).
\paragraph{}Notre objectif est de réaliser une application Android, destinée principalement aux tablettes. Cette application regroupera les jeux, mais il ne redirigerons pas vers une application en Flash mais vers une version Android, les jeux devrons \^{e}tre redéveloppé. Celle-ci sera utilisé pour l'apprentissage des enfants, il faut donc faire attention à l'ergonomie de l'application.
\paragraph{}Nous avons la charge de d'étudier l'actuel portails, concevoir, développer l'application Android. Et comme challenge supplémentaire, une API qui permettrai de construire des jeux sans passer par \java{}.


%%%%%%%%%%%%%%%%%%%%%%%%%%%%%%%%%%%%%%%%%%%%%%%%%%%%%%%%%%% Chap :
\chapter{Ressources et outils utilisés}
%% ************************************************** %
\section{Documentation}
\subsection{\googleDrive}
% TODO rajouter le lien vers l'annexe
Il a été décidé lors de notre première réunion (voir annexe ???/???) que les documents de travails (exemple: résumé de réunion, glossaire, ...) sont a stocker dans un repertoire de \googleDrive{}. Les document de type \og{}\google{}\fg{} sont à privilégier.
\subsection{Latex}
Nous nous sommes servi du langage \LaTeX{} uniquement pour la rédaction de ce rapport et pour notre présentation.
%% ************************************************** %
\section{Développement}
%Pour développer des applications \android{}, \google{} préconise l'utilisation d'\eclipse{} sur \ubuntu{}.
\begin{description}
\item[\os{}: ] Le développement se fait sur différentes distributions \linux{}
\item[\ide{} et language: ] Pour le développement, nous utilisons le SDK de \android{} couplé avec l'\ide{} \eclipse{}. Le language utilisé est \java{}
\item[Gestionnaire de depôts: ] Ce projet étant réalisé en équipe qui ne se voit pas régulièrement. Il a été décidé dans un premier temps (par \responsableProjet{}) d'utiliser un gestionnaire de source. Nous utilisons \github{} à la fois pour les sources du projet mais aussi (dans un second projet \github{}) les sources de notre rapport
\end{description}


%%%%%%%%%%%%%%%%%%%%%%%%%%%%%%%%%%%%%%%%%%%%%%%%%%%%%%%%%%% Chap :
\chapter{Projet}
%%% ************************************************** %
\section{Analyse}
\paragraph{}Après le lancement du projet, il était nécessaire de faire un état des lieux sur pepit.be. Nous avons donc fait une analyse portant sur les exercices se trouvant sur le site web.

La structure des données est la suivante : Niveau -> Thème -> Exercice -> Module

\begin{description}
\item[Niveaux : ] Maternelles, Niveau 1, Niveau 2, Niveau 3, Niveau 4, Niveau 5, Niveau 6, Conjugaison, Table de multiplication, Enseignement spécial, Pour tous et Secondaire.
\item[Thèmes : ] Français, Mathématiques et Divers.
\end{description}

\paragraph{}L'application web utilise de nombreuses images, ce qui pose un problème sous \android, l'application sera de ce faite plus lourde. Ainsi que certains exercices utilisent des animations, ce qui demande une certaine ma\^{i}trise du développement mobile.
\paragraph{}L'utilisation du clavier revient souvent, mais ce-ci n'est pas un problème pour l'application mobile.

\subsection{Formes courantes}
\paragraph{}Deux structures d'exercices reviennent souvent :
\begin{description}
\item[Comparaison : ] entre des images ou des mots.
\item[Question / Réponse : ] avec proposition, phrase avec des morceaux manquants ou encore un calcul sans résultat.
\end{description}

%%% ************************************************** %
\section{Architecture}
%%% ************************************************** %
\section{Maquettes}
\paragraph{}Pour créer une interface mobile, il faut se mettre dans la peau de l'utilisateur, car une application mobile ne se conçoit pas comme un site web ou une application PC par exemple. Celle-ci va être utiliser en tactile et l'écran est plus petit ou de taille variable.

\paragraph{}En suivant l'architecture de l'application, nous avons donc créer l'interface utilisateur. Celle-ci est disponible en annexe, ce n'est pas le design final mais la structure, elle oui.

%%% ************************************************** %
\section{Design}
%%% ************************************************** %
\subsection{Charte graphique}

%%%%%%%%%%%%%%%%%%%%%%%%%%%%%%%%%%%%%%%%%%%%%%%%%%%%%%%%%% Chap :
\chapter*{Bilan et conclusion}
%% ************************************************** %
\section*{Bilan}
???
%% ************************************************** %
\section*{Conclusion}
???

%%%%%%%%%%%%%%%%%%%%%%%%%%%%%%%%%%%%%%%%%%%%%%%%%%%%%%%%%% ANNEXES
\appendix 
\chapter{Références}

\begin{itemize}
\item[Pepit.be :] Site web pepit, portails de jeux éducatifs
\end{itemize}


\end{document}

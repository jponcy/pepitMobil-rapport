\section{\pepit}
\paragraph{}PEPIT est né le 15 mai 2004 à Mouscron en Belgique. Il s'agissait, à l'époque, de créer une plateforme numérique d'échange d'exercices entre 5 écoles primaires et le premier degré d'une école secondaire.
\paragraph{}Le projet a connu différentes évolutions au fil du temps, la troisième version du site est actuellement en ligne. Cette dernière version contient plus de 1250 exercices en ligne, la variété et la qualité de nos exercices sont appréciés les mois "scolaires" par plus de 200.000 "visiteurs uniques".
%% ************************************************** %
\section{Et nous, que vient-on faire là dedans ?}
\paragraph{}Actuellement, le pepit.be est un portail d'accès à des applications Flash, où il est possible de jouer directement sur le site ou de télécharger le jeu sur l'ordinateur (PC ou MAC).
\paragraph{}Notre objectif est de réaliser une application Android, destinée principalement aux tablettes. Cette application regroupera les jeux, mais il ne redirigerons pas vers une application en Flash mais vers une version Android, les jeux devrons \^{e}tre redéveloppé. Celle-ci sera utilisé pour l'apprentissage des enfants, il faut donc faire attention à l'ergonomie de l'application.
\paragraph{}Nous avons la charge de d'étudier l'actuel portails, concevoir, développer l'application Android. Et comme challenge supplémentaire, une API qui permettrai de construire des jeux sans passer par \java{}.
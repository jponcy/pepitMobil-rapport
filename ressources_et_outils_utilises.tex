\section{Documentation}
\subsection{\googleDrive}
Nous utilisons cette plate-forme collaborative, pour une meilleur gestion et partage des fichiers du groupe de travail. Nous privilégions l'utilisation des formats de document \google{} pour une meilleure harmonisation et compatibilité sur le cloud.

Raisons techniques de ce choix :
\begin{itemize}
\item Un seul point de stockage pour le groupe
\item Suite bureautique à disposition
\item Stockage de fichiers, peu importe le type de fichier
\item Plate-forme collaborative (édition simultanée, commentaires, ...)
\end{itemize}

\subsection{Latex}
Nous utilisons \LaTeX{} car \googleDocuments{} ne permet pas de réaliser une bonne mise en page pour un dossier, gr\^{a}ce à \LaTeX{} nous pouvons facilement créer des tables de matières ou des liens hypertextes par exemple, ce qui n'est pas aisé sur \googleDocuments{}.

Nous utilisons \LaTeX{} juste pour le rapport de mi-parcours et le rapport final.

%% ************************************************** %
\section{Développement}
%\begin{description}
%\item[\os{}:] Le développement se fait sur différentes distributions \linux{}
%\item[\ide{} et langage:] Pour le développement, nous utilisons le SDK d'\android{} couplé avec l'\ide{} \eclipse{}. Le langage utilisé est \java{}
%\item[Gestionnaire de depôts:] Ce projet étant réalisé en équipe qui ne se voit pas régulièrement. Il a été décidé dans un premier temps (par \responsableProjet{}) d'utiliser un gestionnaire de sources. Nous utilisons \github{} à la fois pour les sources du projet mais aussi (dans un second projet \github{}) les sources de notre rapport
%\end{description}
Pour développer des applications \android{}, \google{} préconise l'utilisation d'\eclipse{} sur \ubuntu{}.
\subsection{\os{}}
Le développement se fait sur différentes distributions \linux{} (Debian et/ou dérivés).
\subsection{\ide{} et langage}
Pour le développement, nous suivont les préconisations de \google{} (avec une tolérance sur le \os{}). Nous utilisont tous le SDK d'\android{} couplé avec l'\ide{} \eclipse{} Juno. Le langage utilisé est \java{} (majoritairement utilisé pour le développement \android{}).
\subsection{Gestionnaire de dépôts}
Ce projet étant réalisé en équipe qui ne se voit pas régulièrement. Il a été décidé dans un premier temps (par \responsableProjet{}) d'utiliser un gestionnaire de sources. Nous utilisons \github{} à la fois pour les sources du projet mais aussi (dans un second projet \github{}) les sources de notre rapport.
\begin{description}
    \item[rapport:] https://github.com/jponcy/pepitMobil-rapport
    \item[Pepitmobil:] (application principale): https://github.com/pepit/PepitMobil
    \item[Pepit team:] https://github.com/pepit-team
\end{description}
%%% ************************************************** %
\section{Android}
Version min du SDK :
\begin{description}
\item[API : ] 12
\item[Version : ] 3.1 
\item[Codename : ] Honeycomb 
\end{description}

Version pour développement du SDK :
\begin{description}
\item[API : ] 17
\item[Version : ] 4.2 
\item[Codename : ] Jelly Bean
\end{description}

\paragraph{}Nous utilisons la version 3.1 d'Android au minimum, car notre application est orientée tablette.
%% ************************************************** %
\section{Méthodologie}
Le projet ne suit aucune méthode \agile{} à propement parlée mais emprunte un certain nombre d'outils et de \og{}bonnes pratiques\fg{} tel que: un \og{}groupe de travail disposant du pouvoir de décision\fg{}; une \og{}spécification et validation permanentes des exigences\fg{}.

La principale pratique empruntée est celle des \sprint s. Un \sprint{} est une période d'un mois au maximum, au bout de laquelle l'équipe délivre un incrément du produit, potentiellement livrable. Une fois la durée choisie, elle reste constante pendant toute la durée du développement. Un nouveau \sprint{} démarre dès la fin du précédent. Dans le cadre de ce projet, les \sprint s sont d'une durée de 3 semaines.
